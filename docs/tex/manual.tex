\documentclass[12pt]{memoir}
\usepackage{appendix}
\usepackage{cite}
\usepackage{amssymb, amsmath}
\usepackage{pstricks}
\usepackage[cm]{fullpage}
\usepackage{palatino}
\usepackage{setspace}
\epigraphfontsize{\small\itshape}
\setlength\epigraphwidth{8cm}
\setlength\epigraphrule{0pt}
\usepackage{graphics}
\usepackage{endnotes}
\newcommand{\mystery}{{R_{\text{M}}}}
\newcommand{\minecraft}{{\textit{Minecraft}}}
\newcommand{\forge}{{\textit{Forge}}}

\date{\today}
\author{sibomots -- {\tt{sibomots@sibomots.net}}}
\title{ {\textbf{SiboRocAdventure Mod Development}} }

\begin{document}

\chapter{About Time}

%%\epigraph{This is indeed the Golden Age.  The greatest rewards come
%%from gold; by gold love is won; by gold is faith destroyed; by gold is
%%justice bought ; the law follows the track of gold, while modesty
%%will soon follow it when law is gone.}{Propertius}

%%\epigraph{I consider that nothing is more powerful than gold.  By it all
%%things are torn asunder; all things are accomplished.}{Diphilus}

\epigraph{Such is oft the course of deeds that move the wheels of
the world: small hands do them because they must, while the eyes of the
great are elsewhere.}{Elrond}
\epigraph{Go not to the Elves for counsel, for they will say both 
no and yes.}{Frodo}

\section{So, where to begin}

Where to begin is a good question.  First a review of the goal.

\begin{itemize}
\item Make a \forge\footnote{\minecraft\ \forge} mod.
\item Make the mod do something interesting.
\item Use new graphical models and user-interfaces (not just text)
\item Integrate the mod with pre\"existing capabilities of mods.
\end{itemize}

If a goal could be anything more than that, then it might prevent
ever getting started.  But those goals alone are a heavy lift especially
if the person reading this is trying to {\textit{learn how to write 
the code for a \forge\ mod}} for the first time.

The reader may even be somewhat familiar with \minecraft\ and perhaps
even worked {\textit{with}} mods, but not necessarily written any
themselves.  On the other hand, the reader may be experienced enough
with the \forge\ framework to know how to get started and what
steps to take. 

Whatever the case, the author doesn't have much background in
actually {\textit{writing mods}}, but he has studied the source code
of many mods, read the documentation for \forge\ and tried to come
to terms with the amount of architecture and design that needs to 
be applied to make a good mod.

\subsection{Goals}

The goals of the project are to

\begin{itemize}
\item Make a \forge\ mod for the 1.16 version of \minecraft. This goal 
is straight forward.  The requirement is to configure the mod to be
built within the \forge\ 1.16 development version.
\item The mod is going to provide the capabiity to work with metals. Metal
working is an activity that would be categorized depending on the age of
the technology in force.   The goal is to make the mod relateable to
the medieval times but imbued with a large dose of fantasy as to make
it interesting, practical as much as fantasy imaginations are (infinitely),
and related to the examples one might know from experience in literature.
\begin{itemize}
\item A goal is to make new {\textbf{ore}}.  The names of the new
ores are:
  \begin{itemize}
   \item {\textbf{Hematite}} - an ore with iron.
   \item {\textbf{Corundum}} - an ore with traces of metal, but it's use is an abrasive.
   \item {\textbf{Chromium}} - a very hard (brittle) metal when refined from ore.
   \item {\textbf{Sibomium}} - an ore that when refined has magical effect on the metal it alloys with.
   \item {\textbf{Glacial Till}} - an mineral rich gravel for construction.
   \item And others to be determined\ldots
  \end{itemize} 
\item A goal is to make ores harvestable.  Biomes will generate
during world setup and these ore blocks will be populated in the world.
\item A goal is of the mod is these ores are ingredients for alloys
with other Vanilla ores, the result of which is material to use for fabricating
special weapons and armor.  It can also be rendered into other
forms for jewlery and magic items.  The result of such alloying recipe 
produces ingots of various categories.
\item Add recpie for how a new ore can be alloyed with other ores
in a special (new) type of furnace designed to mix ores.  
\item Add a new furnace type. It will be a primitive type that 
requires {\textbf{bellows}}, an energy source from a {\textbf{water wheel}}
(to rotate equipment, provide mechanical actuation of the mechanisms), 
a requirement for indoor use to be ventilted {\textbf{chimney}},
\item Items that can be built from alloys from the refined ore:
 \begin{itemize}
    \item A {\textbf{anvil}}  (used to make tools and forge parts)
    \item A leather {\textbf{bellows}} (used to increase heat in furnacce)
    \item A {\textbf{crowbar}} (used to split blocks of rocky substance)
    \item A {\textbf{drill bit}} (used to make tools and parts)
    \item A {\textbf{friction bearing}} (used to fabricate machines)
    \item A {\textbf{flue}} section for chimney (used to build furnace)
    \item A {\textbf{blast furnace}} (used to smelt ore and make alloys)
    \item A {\textbf{grinding}} wheel (used to make tools and parts)
    \item A {\textbf{massey hammer}} (used to forge metal and make tools)
    \item A leather {\textbf{pulley belt}} (used to fabricate machines)
    \item A {\textbf{round bar}} stock (used to make tools and parts)
    \item A {\textbf{square bar}} stock (used to make tools and parts)
    \item A {\textbf{tongs}} (used to make tools and parts)
    \item A {\textbf{water wheel}} (used to activate machines)
    \item A {\textbf{wooden gear}} (used to fabricate machines)
    \item A {\textbf{wooden pinion}} (used to fabricate machines)
    \item A {\textbf{grease}} (used to maintain machines and tools)
    \item And others to be determined\ldots
 \end{itemize}

Weapons that cxan be built using metal alloys refined from these ores:

   \begin{itemize}
      \item A {\textbf{bastard sword}}
      \item A {\textbf{battle axe}}
      \item A {\textbf{chain}}
      \item A {\textbf{dagger}}
      \item A {\textbf{morning star}}
      \item A {\textbf{pike}}
      \item A {\textbf{short-sword}}
      \item And others to be determined\ldots
   \end{itemize}
\end{itemize}
\item Add the secret functionality that pulls the ores, items, objects, etc\ldots together into
 a system that has a purpose, a point.  
\item Start with something simple in terms of a mod so that the basics can be planned, understood,
implemented and run with some degree of satisfaction.  
\end{itemize}

\chapter{The Development Environment}
\section{Etc\dots}

TBD

\chapter{Mod Design}
\section{Etc\ldots}
Section(s) for how to implement the mod.


\chapter{Installation}
\section{Etc\dots}
TBD

\chapter{Play}
\section{Etc\dots}
TBD


\chapter{Configuration}
\section{Etc\dots}
Once the mod is installed there are configuration options that
can be set by the user.

TBD

\chapter{Integration with Vanilla Version}
\section{Etc\dots}

There are elements of the mod that require integration with a set of
Vanilla features of the game.

TBD

\end{document}
